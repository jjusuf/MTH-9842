\documentclass{article}

%% Language and font encodings
%\usepackage[english]{babel}
%\usepackage[utf8x]{inputenc}
%\usepackage[T1]{fontenc}

%% Sets page size and margins
\usepackage[margin=1in]{geometry}

%% Useful packages
\usepackage{amsmath}
\usepackage{graphicx}
\usepackage[colorinlistoftodos]{todonotes}
%\usepackage{mathrsfs}
\usepackage{amsfonts,amssymb}
\usepackage[colorlinks=true, allcolors=blue]{hyperref}
\usepackage{parskip}
\setlength{\parindent}{0pt}

\title{Homework 2}
\author{Group L\\
Yang Li, Jimmy Jusuf, Zhirong Zhang, Tiangang Zhang}

\begin{document}
\maketitle


\section{Problem 1}
\begin{eqnarray*}
L(x,y,\lambda)=(x-a)^2+(y-b)^2-\lambda_1 x+\lambda_2 (x-1)-\lambda_3 y +\lambda_4 (y-1)
\end{eqnarray*}
Then local minimizer should subject to

\begin{equation}  
\left\{  
             \begin{array}{ll}
             2(x^*-a)-\lambda_1^*+\lambda_2^*=0\\  
             2(y^*-a)-\lambda_3^*+\lambda_4^*=0\\
             \lambda_1^* x^*=0\\
             \lambda_2^*(x^*-1)=0\\
             \lambda_3^* y^*=0\\
             \lambda_4^*(y^*-1)=0\\
             \lambda_i^* \geq 0 & \textnormal{for~} i=1,2,3,4
             \end{array}  
\right.  
\end{equation}  

Solve the equations above, the solution is
\begin{equation}  
x^*=
\left\{  
             \begin{array}{ll}
             0 & a\leq 0\\
             a & a \in (0,1) \\
             1 & a \geq 1
             \end{array}  
\right.  
\end{equation} 
\begin{equation}  
y^*=
\left\{  
             \begin{array}{ll}
             0 & b\leq 0\\
             b & b \in (0,1) \\
             1 & b \geq 1
             \end{array}  
\right.  
\end{equation}  

\section{Problem 2}
\subsection{Solution 1}
We replace constraints of the form
\begin{equation} |x_i| \leq b_i \end{equation}
with
\begin{equation}
\begin{split}
	x_i \leq b_i \\
	-x_i \leq b_i
\end{split}
\end{equation}

To deal with the absolute value in the objective function, we introduce a split variable
\[ y = y^+ - y^- \textnormal{~with constraints~} y^+, y^- \geq 0 \]

Note that $|y| = y^+ + y^-$ because by construction, for any value of $y$, $y^+ * y^- = 0,   y^+, y^- \geq 0$.

Thus,
\begin{gather}
x_2 - 10 = y \implies x_2 = y + 10 = y^+ - y^- + 10 \\
|x_2 - 10| = |y| = y^+ + y^- \\
\end{gather}

By change of variables and substitutions above, the nonlinear problem
\[ \min 2 x_1 + 3|x_2 - 10| \quad \textnormal{subject to~} |x_1 + 2| + |x_2| \leq 5 \]
can be rewritten as LP problem
\begin{equation}
\min \, 2x_1+3(y^+ + y^-), \textnormal{ subject to}
\left\{
    \begin{array}{l}
    x_1 + 2 + y^+ - y^- + 10 \leq 5 \\
    -x_1 - 2 + y^+ - y^- + 10 \leq 5 \\
    x_1 + 2 - y^+ + y^- - 10 \leq 5 \\
    -x_1 - 2 - y^+ + y^- - 10 \leq 5 \\
     y^+, y^- \geq 0
    \end{array}
\right.
\end{equation}

\subsection{Solution 2}
Let $x_2-10=x^+-x^-,\, x^+,x^-\geq 0$, use the property of absolute value inequalities, the problem become
\begin{equation}  
\min \, 2x_1+3|x^+-x^-|, \textnormal{ subject to}
\left\{  
             \begin{array}{ll}
             x_1+2+10+x^+-x^-\leq 5\\
             -x_1-2+10+x^+-x^-\leq 5\\
             x_1+2-10-x^++x^-\leq 5\\
             -x_1-2-10-x^++x^-\leq 5\\
             x^+,x^-\geq 0
             \end{array}  
\right.  
\end{equation}  

Notice that 
\begin{equation}  
2x_1+3(x^++x^-) \geq 2x_1+3|x^+-x^-|
\end{equation}  
now we consider 
\begin{equation}  
\min \,2x_1+3(x^++x^-) , \textnormal{ subject to}
\left\{  
             \begin{array}{ll}
             x_1+2+10+x^+-x^-\leq 5\\
             -x_1-2+10+x^+-x^-\leq 5\\
             x_1+2-10-x^++x^-\leq 5\\
             -x_1-2-10-x^++x^-\leq 5\\
             x^+,x^-\geq 0
             \end{array}  
\right.  
\end{equation}  
Let $x_1^*,x^{+*},x^{-*}$ be the minimizer of the problem above, the next step is to prove that  $x_1^*,x^{+*},x^{-*}$ is the minimizer of the original problem, we prove by contradiction.
If there is $x_1',x_2'$ subject to the conditions and 
\begin{equation}  
2x_1'^*+3|x_2'-10| < 2x_1^*+3(x^{+*}+x^{+*})
\end{equation}  
Let 
\begin{eqnarray*}  
x_{1\star}=x_1'\\
x^{+}_\star=\{x_2'-10\}^+\\
x^{^-}_\star=\{-x_2'+10\}^+
\end{eqnarray*}  
Then 
\begin{equation}  
2x_{1\star}+3(x^{+}_\star+x^{^-}_\star) < 2x_1^*+3(x^{+*}+x^{+*})
\end{equation}  
Contradiction!

Recall equation(5) we know that
\begin{equation}  
2x_1'^*+3|x_2'-10| = 2x_1^*+3(x^{+*}+x^{+*})
\end{equation}  
So we can always use the minimizer of problem (6) to get the minimizer and minimum value of the original problem. Then problem has been transferred into a LP
\begin{equation}  
\min \,2x_1+3(x^++x^-) , \textnormal{ subject to}
\left\{  
             \begin{array}{ll}
             x_1+2+10+x^+-x^-\leq 5\\
             -x_1-2+10+x^+-x^-\leq 5\\
             x_1+2-10-x^++x^-\leq 5\\
             -x_1-2-10-x^++x^-\leq 5\\
             x^+,x^-\geq 0
             \end{array}  
\right.  
\end{equation}  

\section{Problem 3}
\newcommand{\ccy}[1]{currency#1}

There are $N$ currencies, index $i=1,\dots,N$.

We start with $B$ units of currency 1 and we want to maximize the number
of units of currency $N$ through a sequence of trades. One unit of 
currency $i$ can be exchanged for $f_{ij}$ units of currency $j$.

Decision variables: let $x_{ij} =$ amount of currency $i$ to be converted 
into currency $j$. Clearly $\sum_{j\not= 1}x_{1j}\leq B$. We also have 
constraint $\sum_{j\not= 1}x_{1j}\leq u_1$, since we exchange out of 
currency 1 exactly once into currency $j$ ($j \not= 1,N$), 
where $u_i$ is the limit on the total amount of currency $i$ that can be
traded. If we exchange out of currency 1 into currency $N$ through 
intermediate currency $j$, the trading limit becomes
$(\sum_{j \not= 1,N}x_{ij} + \sum_{k \not= 1,N}x_{jk}) \leq u_j$, for $j \not= 1,N$.

Objective function: maximize $\sum_{i \not= N} f_{iN} * x_{iN}$, essential
maximizing the total inflow into currency $N$ from all other currencies. 
The no-arbitrage condition 
$f_{i_1 i_2}\,f_{i_2 i_3}\dots f_{i_{k-1} i_k}\,f_{i_k i_1}\leq 1 $ means
that if we start with any currency, continually exchange one currency into another
currency, and eventually back into the original currency, we can never end up
with more units of the original currency, i.e. if we start with 1 unit, at
most we get back 1 unit, or could be less than 1, after exchanging into a series of
other currencies. It also implies there is no trading ``loops'' or trading 
sequences in which one can increase currency count in this economy. 
The trade from one currency to another currency happens at most for one time.
To prevent exchanging out of currency $i$ into the same currency, 
we impose constraint $x_{ii}=0$ for $i=1,\dots,N$. To
``force'' trading progress towards currency $N$ and avoid ``loops'', we let 
$x_{ij}\geq 0$ for $i=1,\dots,N, j>i$ and $x_{ij}=0$ for $i=1,\dots,N, j<i$,
basically $x_{ij}$ is an upper triangular square matrix with zero diagonal.

% According to no-arbitrage condition, there is no need to trade a 
% currency back to the currency it traded from. 
In summary, the LP formulation of this problem is
\begin{equation}  
\max \,\sum_{i \not= N} f_{iN} * x_{iN}, \text{ subject to}
\left\{  
             \begin{array}{ll}
             \sum_{j\not= 1}x_{1j}\leq B,\\
             \sum_{j\not= 1}x_{1j}\leq u_1,\\
             \sum_{j\not= 1, i}x_{ij}\leq \sum_{k\not=i,N}f_{ki} x_{ki},& \textnormal{for~} i\not= 1,N\\
             \sum_{j\not= 1, i}x_{ij}\leq u_i,& \textnormal{for~} i\not= 1,N\\
             x_{ij}\geq 0, & \textnormal{for~} i\not= j,i\not=N
             \end{array}  
\right.  
\end{equation}  
\end{document}
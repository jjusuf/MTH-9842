\documentclass[fleqn]{article}

%% Language and font encodings
%\usepackage[english]{babel}
%\usepackage[utf8x]{inputenc}
%\usepackage[T1]{fontenc}

%% Sets page size and margins
\usepackage[margin=1in]{geometry}

%% Useful packages
\usepackage{amsmath}
\usepackage{graphicx}
\usepackage[colorinlistoftodos]{todonotes}
%\usepackage{mathrsfs}
\usepackage{amsfonts,amssymb}
%\usepackage[colorlinks=true, allcolors=blue]{hyperref}
\usepackage{parskip}
\usepackage{ragged2e}

\newcommand{\mat}[1]{\mathbf{#1}}
\newcommand{\tr}{^\intercal}
\newcommand{\real}{\mathbb{R}}
\newcommand{\vect}[1]{\begin{pmatrix}#1\end{pmatrix}}
\newcommand{\nablax}{\nabla_{\!\mat x}}

\DeclareMathOperator{\Lagr}{\mathcal{L}}

%\setlength{\parindent}{0pt}

\title{Homework 2}
\author{Group L\\
Yang Li, Jimmy Jusuf, Zhirong Zhang, Tiangang Zhang}

\begin{document}
\maketitle


\section{Problem 1}

\section{Problem 2}

Let
\begin{equation}
\mat A = \begin{pmatrix}
  13 & 12 & -2 \\
  12 & 17 & 6 \\
  -2 & 6 & 12 \\
  \end{pmatrix} = \mat A\tr,\quad 
\mat a = \begin{pmatrix} -22 \\ -14.5 \\ 13 \\ \end{pmatrix},\quad
b = 1.
\end{equation}
\begin{equation}
\textnormal{objective function to be minimized }f:\real^3\rightarrow\real,
\enskip f(\mat x)=\frac12 \mat x\tr \mat A \mat x + \mat a\tr x + b,
\end{equation}
\begin{equation}\label{eq3}
\textnormal{subject to } {-1}\leq x_i\leq 1, i=1,2,3
\enskip\textnormal{ or } \begin{pmatrix} -1\\-1\\-1 \end{pmatrix}
  \leq \mat x\leq \begin{pmatrix} 1\\1\\1 \end{pmatrix},
  \mat x\in\real^3.
\end{equation}

Write inequality constraints \eqref{eq3} as 
$h_1(\mat x)=-\mat x+\begin{pmatrix} -1\\-1\\-1 \end{pmatrix}\leq 0$
and
$h_2(\mat x)=\mat x+\begin{pmatrix} -1\\-1\\-1 \end{pmatrix}\leq 0$,
$h_1, h_2:\real^3\rightarrow\real$.
Note that
$\nablax f(\mat x)=\mat A\mat x + \mat a=
  \vect{21\\14.5\\-11} + \vect{-22\\14.5\\13} = \vect{-1\\0\\2}$,
$\nablax h_1(\mat x)=\vect{-1\\-1\\-1}$, 
$\nablax h_2(\mat x)=\vect{1\\1\\1}$.

Let $
\Lambda_1=\begin{pmatrix}\lambda_1\\ \lambda_2\\ \lambda_3\end{pmatrix},\enskip 
\Lambda_2=\begin{pmatrix}\lambda_4\\ \lambda_5\\ \lambda_6\end{pmatrix}$ be the Lagrange multipliers
for constraints $h_1(\mat x)$ and $h_2(\mat x)$ respectively.

Define Lagrangian
\begin{gather}
\Lagr(\mat x, \mat\Lambda)=f(\mat x)+ \Lambda_1\tr\cdot h_1(\mat x) + 
  \Lambda_2\tr\cdot h_2(\mat x)
\end{gather}

%    Given $\mat x^* = \vect{1 \\ 0.5 \\ -1}$
%and $\lambda =
%\begin{pmatrix} \lambda_1 - \lambda_2 \\ \lambda_3 - \lambda_4 \\ \lambda_5 - \lambda_6 \\ \end{pmatrix}$

For $\mat x^*$ to be a local minimum, we want to show
the following KKT Necessary Conditions are met.
\begin{gather}
\nablax \Lagr(\mat x^*, \Lambda^*) = 
  \nablax f(\mat x) + (\Lambda_1^*)\tr\cdot\nablax h_1(\mat x)
  + (\Lambda_2^*)\tr\cdot\nablax h_2(\mat x) = 0 \label{eq5}\\
%  &= (\mat A\mat x + \mat a) + (\Lambda_1^*)\tr\cdot\vect{-1\\-1\\-1}
%    + (\Lambda_2^*)\tr\cdot\vect{1\\1\\1}
(\Lambda_1^*)\tr \cdot h_1(\mat x^*) = 0 \label{eq6}\\
(\Lambda_2^*)\tr \cdot h_2(\mat x^*) = 0 \label{eq7}\\
\Lambda_1^* \geq 0 \label{eq8}\\
\Lambda_2^* \geq 0 \label{eq9}
% d^T \nabla_{xx}^2 L(x^*, \lambda^*) d > 0
\end{gather}

Plugging in $\mat x^* = \vect{1 \\ 0.5 \\ -1}$,\enskip
\begin{gather}
\nabla_{\!\mat x} \Lagr(\mat x^*, \Lambda^*) = \vect{-1\\0\\2}
  +\vect{-\lambda_1\\-\lambda_2\\-\lambda_3}
  +\vect{\lambda_4\\ \lambda_5\\ \lambda_6}=0 \tag{\ref{eq5}a}\label{eq5a}\\
\vect{\lambda_1&\lambda_2&\lambda_3} \vect{-2\\-1.5\\0} = 0
  \implies \lambda_1=\lambda_2=0 \tag{\ref{eq6}a}\label{eq6a}\\
\vect{\lambda_4&\lambda_5&\lambda_6} \vect{0\\-0.5\\-2} = 0
  \implies \lambda_5=\lambda_6=0 \tag{\ref{eq7}a}\label{eq7a}
%\Lambda_2^* \cdot h_2(\mat x^*) = 0 \\
%\Lambda_1^* \geq 0 \\
%\Lambda_2^* \geq 0 \\
\end{gather}
The \emph{complementary slackness condition} forces
$\lambda_1,\lambda_2,\lambda_5,\lambda_6$ to be zero as
$h_1(x_1), h_1(x_2), h_2(x_2), h_2(x_3) \leq 0$ in
equations \eqref{eq6a} and \eqref{eq7a}.
Back-substitution into \eqref{eq5a} yields $\lambda_4=1$ and 
$\lambda_3=-2$. All the $\lambda$'s satisfy conditions \eqref{eq8}
and \eqref{eq9}.

We have shown the KKT Necessary Conditions are met and there is
a unique vector of Lagrange multipliers for all the constraints.

%From the constraints and the given solution, we have
%\begin{gather}
%\lambda_2 = \lambda_3 = \lambda_4 = \lambda_5 = 0 \\
%\lambda_1 = 1 \\
%\lambda_6 = 2
%\end{gather}

%As all the leading principal minor of \(A\) is positive, the second
%order condition is met.

\end{document}